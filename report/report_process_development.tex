\section{Verfahrensentwicklung}

% a) Kalibrierung und Sensitivitätsanalyse
Das im vorhergehenden Kapitel entwickelte Tool soll im Folgenden zunächst am Beispiel einer Flächenanalyse der Stadt Karlsruhe getestet werden, da im Gegensatz zur Verkehrssituation für diesen Fall genaue Vergleichsdaten vorliegen. Es werden die für die Untersuchung entscheidenden Parameter ermittelt und deren Einfluss auf die untersuchten Größen untersucht. Im Anschluss daran wird in einem zweiten Abschnitt ein standardisiertes Vorgehen vorgestellt, welches zur systematischen Verkehrsanalyse verschiedener Städte und Gebiete verwendet werden soll.\\

\subsection{Kalibrierung und Sensitivitätsanalyse}

Das entwickelte Tool wird nun zunächst auf das Gebiet des Stadtgebiets Karlsruhe angewendet. Der Stadtkreis Karlsruhe umfasst laut Statistischem Landesamt Baden-Württemberg (Stand 2015) eine Fläche von rund \num{17.346} \si{\hectare} \cite{StatBaWu_Flaeche}, die Einwohnerzahl beläuft sich auf \num{307755} \cite{StatBaWu_Einw}.\\
Es gilt zu zeigen, dass das Verfahren die reale Flächenaufteilung im Stadtgebiet abbilden kann. Hierbei verfolgt google maps eine eigene Klassifizierung der Flächennutzungen, die sich nicht mit den sogenannten "\textit{tatsächlichen Nutzungsarten}" der  \textit{Arbeitsgemeinschaft der Vermessungsverwaltungen der Länder der Bundesrepublik Deutschland (AdV)}, welche die Grundlage aller deutschen Liegenschaftskataster bilden \cite{advnutz} . Als Vergleichsbasis wird der Anteil an Naturfläche gewählt, da dieser in den Klassifizierungen der AdV und in denen von google maps ähnlich definiert ist.\\ 
\newline
Zunächst wird untersucht, welche Zoomstufe in google maps sinnvollerweise ausgewählt werden muss, um das Stadtgebiet darstellen und analysieren zu können. Bei der Entscheidung, mit welcher Zoomstufe gearbeitet werden soll, gilt es, zwischen den Genauigkeitsanforderungen  der Daten und einer akzeptablen, zu verarbeitenden Datenmenge abzuwägen. Hierbei gelten die folgenden beiden Einschränkungen:\\
\begin{itemize}
\item Einerseits sollten das erzeugte Analysegebiet möglichst wenig über die Grenze des Stadtgebiets hinausragen, da die Analyse der Flächennutzung sonst verfälscht wird. Im vorliegenden Fall kann die minimale Zoomstufe damit zu 12 festgelegt werden, wie Abbildung \ref{fig:Stadtgebiet_KA} zeigt, da in diesem Fall das gesamte Stadtgebiet auf nur einer Kachel dargestellt werden kann.
\item Auf der anderen Seite wird die maximale Zoomstufe von google je nach Datengrundlage vorgegeben, in Karlsruhe ist die maximale Zoomstufe mit 21 gegeben. Wie Abbildung \ref{fig:Schloss_KA} am Beispiel des Karlsruher Schlosses darstellt, führt dies auf eine Darstellung auf Gebäudeebene. Eine solch hohe Diskretisierung liefert zwar eine äußert präzise Darstellung der städtischen Flächen, ist aus der Gründen der Speicher- und Rechenaufwandes nicht sinnvoll, da zur Darstellung des Stadtgebietes in diesem Fall mehrere 10.000 Einzelkacheln analysiert werden müssten.
\end{itemize}
Um das gesamte Stadtgebiet mit hoher Präzision abbilden zu können und gleichzeitig eine adaptive Anpassung der Kachelauswahl an die Stadtgrenze zu ermöglichen, wird in der vorliegenden Kalibrierung wird mit einer (vergleichweise hohen) Zoomstufe von 17 gearbeitet. Auch hier wird für das Stadtgebiet Karlsruhe bereits die Analyse von knapp 1500 Einzelkacheln notwendig, weswegen für die spätere Anwendung eine geringere Zoomstufe empfohlen wird.\\
\newline
%
\begin{figure}
  \centering
    \includegraphics[width=0.55\textwidth]{images/3_Stadtgebiet_KA_zoom12.png}
    \caption{Darstellung der Gemarkungsgrenze des Stadtkreises Karlsruhe [Quelle: google maps, Zoomstufe 12]}
    \label{fig:Stadtgebiet_KA}
\end{figure}
%
\begin{figure}
  \centering
    \includegraphics[width=0.6\textwidth]{images/3_KA_Schloss_zoom21.png}
    \caption{Darstellung des Karlsruher Schlosses bei höchster verfügbarer Zoomstufe [Quelle: google maps, Zoomstufe 21]}
    \label{fig:Schloss_KA}
\end{figure}
%
Da google maps die wählbaren Zoomstufen nicht mit einem festen kartographischen Maßstab verknüpft, muss dieser durch eigene Abstandsmessungen ermittelt werden. Bei Zoomstufe 17 besitzt eine der erzeugten quadratischen Einzelkacheln beispielsweise eine Kantenlänge von ca. \num{500} \si{\metre}, wie in Abbildung \ref{fig:Zoomvgl} zu sehen ist.\\
%
\begin{figure}
  \centering
    \includegraphics[width=0.6\textwidth]{images/3_Zoomvergleich_KA.png}
    \caption{Vergleich der Kachelabmessungen bei verschiedenen Zoomstufen}
    \label{fig:Zoomvgl}
\end{figure}
%
Im Weiteren wird die Flächenanalyse nun für verschiedene Anzahlen an Kacheln durchgeführt, um zu untersuchen, wie sich die Flächenanteile mit zunehmendem Einzugsgebiet der Analyse verändern. Die Ergebnisse der Flächenanalyse für die Eingabe an Tiles je Richtung zwischen 12 und 20 ist in Abbildung \ref{fig:Zoomvgl} dargestellt.\\
%
\newline
\begin{figure}
  \centering
    \includegraphics[width=0.92\textwidth]{images/3_Kachelvergleich_KA.png}
    \caption{Ergebnisse der Flächenanalyse für den Stadtkreis Karlsruhe für verschiedene Kachelanzahlen}
    \label{fig:Kachel_vgl}
\end{figure}
%
Mit steigender Kachelzahl wird immer mehr des Stadtrandes und des Umlandes in die Analyse einbezogen, die Anteile von Verkehrsflächen und \textit{man-made area} an der Gesamtfläche sinken ab zu Gunsten der Naturflächen, welche bei  \((n:=15)\) (Kantenlänge des Analysequadrates ca. \num{14.5} \si{\kilo\metre}) und  \((n:=16)\) (Kantenlänge des Analysequadrates ca. \num{15.5} \si{\kilo\metre}) mehr als \num{60} \% einnehmen. Die zuletzt erwähnten Kachelanzahlen erzeugen ein Analysequadrat, welches in seinen Ausdehnungen in der Größenordnung des Stadtgebiets liegt. Für eine vollkommene Umschließung des Stadtgebiets (maximale Nord-Süd-Ausdehnung ca. \num{16.5} \si{\kilo\metre}), maximale Ost-West-Ausdehnung ca. \num{19.0} \si{\kilo\metre}) muss für die Anzahl an Kacheln \((n:=20)\) gewählt werden. \\
\newline
Allerdings muss erwähnt werden, dass die in Abbildung \ref{fig:Stadtgebiet_KA} dargestellte Gemarkung durch eine solche quadratische Analysefläche nur unzureichend angenähert werden kann. Dies zeigt ein Vergleich der Analyseergebnisse mit den Werten des Statistischen Landesamtes: hier wird der Anteil von Naturflächen mit ca. \num{52} \% angegeben, während die Analyse ca. \num{70} \% berechnet. Dies kann darauf zurückgeführt werden, dass bestimmte Bereiche wie beispielsweise große natürliche und landwirtschaftliche Bereiche in der Analyse erscheinen, obwohl sie nicht zum Stadtgebiet Karlsruhe gehören.\\
%
\newline
\begin{figure}
  \centering
    \includegraphics[width=0.92\textwidth]{images/3_Kachelvergleich_KA_skip.png}
    \caption{Ergebnisse der Flächenanalyse für den Stadtkreis Karlsruhe mit und ohne Kachelauswahl im Vergleich zu den statistischen Daten}
    \label{fig:Kachel_skip_vgl}
\end{figure}
%
%
\newline
\begin{figure}
  \centering
    \includegraphics[width=0.92\textwidth]{images/3_google_Einstufung_Nutzung.png}
    \caption{Darstellung einer Straße in der Web-Ansicht (links) und Einstufung als man-made area nach dem Download (rechts)}
    \label{fig:versch_nutz_einstuf}
\end{figure}
%
Eine bessere Approximation kann durch gezielte Auswahl bzw. gezieltes Ausschließen einzelner Kacheln erreicht werden. Im Folgenden werden daher aus der Berechnung, die auf einer Kachelanzahl \((n:=20)\) basiert, diejenigen Kacheln ausgeschlossen, die außerhalb des Stadtgebietes liegen, sodass die übrigen die Stadtgrenze bestmöglich approximieren. Dadurch verringert sich die Gesamtzahl der analysierten Kacheln von  ursprünglich \(1521\) auf nun \(680\). Das Resultat der Berechnung mit selektiertem Analysegebiet ist in Abbildung \ref{fig:Kachel_skip_vgl} den Ergebnissen der Berechnung mit vollen \(1521\) Kacheln sowie den aufbereiteten Daten des Landesamtes gegenübergestellt. Hierbei ist zu beachten, dass im Gegensatz zur bisherigen Benennung die Kategorien \textit{roads}, \textit{highway} sowie \textit{transit} in dieser Darstellung zur Oberkategorie \textit{traffic} zusammengefasst wurden, um eine Vergleichbarkeit der Berechnungsergebnisse mit den statisitschen Daten, welche einer anderen Klassifizierung folgen, zu ermöglichen. Während die Flächenanteile der Kategorie \textit{man-made} in der Berechnung um weniger als  \num{2} \% von denen des statistischen Landesamtes abweichen, zeigt sich bei Verkehrsflächen respektive Naturflächen eine Abweichung von rund \num{5} bis \num{6} \%. Letzteres kann unter anderem auf die unterschiedlichen Einstufungen der Flächen als Siedlungs- oder als Verkehrsfläche zurückgeführt werden. Ein Beispiel hierfür ist der Bereich um das Ettlinger Tor in Karlsruhe, der in Abbildung \ref{fig:versch_nutz_einstuf} dargestellt ist: Während in der Webansicht von google maps (links) in Kachelmitte eine in Nord-Süd-Richtung verlaufende Straße zu erkennen ist, wird diese von google tatsächlich, wie sich bei der Analyse (rechts) zeigt, als \textit{man-made area} eingestuft.\\

Anhand des Beispiels Karlsruhe konnte damit gezeigt werden, dass das entwickelte Verfahren die reale Flächenverteilung gut annähern kann. 

\subsection{Leitfaden zur Verkehrsanalyse}
Nachdem anhand des Beispiels Karlsruhe gezeigt worden ist, dass das entwickelte Verfahren zur Flächenanalyse die tatsächliche Flächennutzungen sehr gut darstellen kann, soll dessen Anwendung im weiteren Verlauf der Untersuchung bei einer Auswahl von Städten durchgeführt werden.
\begin{enumerate}
\item Festlegen des Zentrums für die Kachelauswahl:
\item Festlegen der Zoomstufe:
\item Einstellung Anzahl Tiles:
\item Auswahl zu überspringende Tiles:
\end{enumerate}


Um eine größere Menge verschiedener Städte möglichst schnell auf ihre verkehrliche Situation zu untersuchen, ist es sinnvoll, ein standardisiertes Vorgehen für die Erstellung einer Verkehrsuntersuchung festzulegen. Die Vorbereitung der Analyse läuft dabei wie folgt ab:

