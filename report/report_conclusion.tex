\section{Fazit und Ausblick}

Diese Arbeit beschreibt die Entwicklung eines Programms zum Auslesen der Verkehrszustände in BingMaps und der Bodenbedeckungsdaten in GoogleMaps, sowie deren Analyse und Integration in einen Stauindex.\\
Das Programm liest erfolgreich auf Basis eines anzugebenden Mittelpunktes, der Zoomstufe und der Anzahl an Kachellagen um den Mittelpunkt die Anzahl der Pixel aus, die mit den jeweiligen Verkehrszuständen bzw. Bodenbedeckungsdaten belegt sind.\\
Weiterhin wird eine Möglichkeit aufgezeigt, diese Daten in einen Stauindex zu integrieren und diesen zeitlich zu mitteln. Auch wenn die Ergebnisse auf eine mögliche Vergleichbarkeit Stauindizes verschiedener Städte hindeuten, ist jedoch eine tiefergehende Untersuchung und insbesondere eine größere Datenbasis erforderlich, um diese Frage abschließend zu beantworten. Hierzu sollten möglichst viele Städte mit unterschiedlichem Verkehrsfluss in die Untersuchung einbezogen werden, um eine wirkungsvolle Kalibrierung durchführen zu können. Es wird hierbei explizit darauf hingewiesen, dass es nicht zwingend notwendig oder sinnvoll sein muss, einen einzelnen Paramatersatz für alle Städte zu entwickeln. Es sollte stattdessen vielmehr geprüft werden, ob eine Klassifizerung oder Gruppierung zweckmäßig ist, wobei dann für jede Klasse ein eigener Parametersatz bestimmt werden kann. Auch in diesem Fall wird jedoch weiterhin eine sachkundige Auswahl der Parameter für das vorliegende Problem notwendig sein.\\

Des Weiteren kann es je nach Zielen und Anforderungen des Anwenders auch notwendig werden, weitere Datenquellen als die bisher genutzten in das Verfahren einzubinden. Hierzu zählen einerseits die unter der \textit{creative-commons} Lizenz freizugänglichen Daten der OpenStreetMap, welche insbesondere zur exakten Abbildung des Straßennetzes herangezogen werden können. Ihre Stärke liegt in der meist hohen Präzision und weitreichenden Verfügbarkeit in weiten Teilen der Erde. Um die Datenbasis in Bezug auf die aktuellen Verkehrs- und Staudaten zu vergrößern, bieten sich im Wesentlichen kommerzielle Datenanbieter an. Gegen entsprechende Bezahlung kann der Nutzer bei diesen weitergehende Verkehrsdaten erhalten, ohne diese, wie im vorliegenden Verfahren getan, aus den Karten mittels Bildanalyse zu extrahieren. Eine höhere Genauigkeit der berechneten Verkehrsanteile ist daher zu erwarten.\\
Eine weitere offene Frage betrifft die Vereinbarkeit der vorgestellten Methodik des Auslesens der Daten mit den Geschäftsbedingungen der Kartenanbieter. Hier zeigen sich Konflikte und eine abschließende Klärung dieser Frage scheint dringend erforderlich.\\

In jedem Fall besteht neben den technischen Weiterentwicklungen besonderes Interesse an den Erkenntnissen aus weitergehenden Untersuchungen verschiedenster Städte, Regionen und Untersuchungsgebiete, die die Anwendbarkeit des entwickelten Verfahrens zeigen.
