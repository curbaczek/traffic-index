\section{Motivation und Zielsetzung}

Die Untersuchung der Verkehrssituation eines (Siedlungs-)Gebietes stellt eine zentrale Aufgabe der strategischen Verkehrsplanung dar. Mögliche Zielsetzungen können hierbei die reine Quantifizierung der aktuellen Verkehrsverteilung, die Entwicklung einer Verkehrsprognose oder die Bewertung einer verkehrlich bedeutsamen (Bau-)Maßnahme sein. Um eine verlässliche Aussage über die Belastungen auf den Verkehrsnetzen eines Siedlungsraumes zu ermitteln, ist die Entwicklung eines Verkehrsmodells heute gängige Praxis in der strategischen Verkehrsplanung.\\
\newline
Bevor eine entsprechend aufwendige und teils langwierige Entwicklung eines Verkehrsmodells in Angriff genommen werden kann, ist eine ausführliche Voruntersuchung notwendig und es bedarf einer umfangreichen Grundlage an Strukturdaten, Verhaltensdaten, Verkehrskennzahlen und vielem mehr. In diesem Stadium der Untersuchung fehlt dem Planer bisher ein wirkungsvolles Tool, welches es ihm ermöglicht, eine Verkehrssituation schnell und automatisiert erfassen zu können.\\
\newline
Die vorliegende Seminararbeit setzt sich als Ziel, ein solches Tool zu entwickeln und dessen Möglichkeiten und Grenzen zu untersuchen. Hierzu soll alleine auf Basis öffentlich zugänglicher online-Kartendienste eine Routine entwickelt werden, welche die Flächennutzung, die aktuelle Verkehrssituation sowie deren zeitliche Veränderung eines Untersuchungsgebietes automatisiert ermittelt.\\
\newline
Im folgenden Kapitel wird zunächst auf die technische Umsetzung und Implementierung des Tools eingegangen. Es werden die Vor- und Nachteile verschiedener Kartendienste genannt und deren Eignung als Grundlage für die vorliegende Untersuchung ausgelotet. Des Weiteren wird beschrieben, wie die gesammelten Daten aus den Kartenelementen gewonnen, analysiert und schließlich visualisiert werden können. Das entwickelte Verfahren wird im daran anschließenden Kapitel auf ein Beispiel angewendet, um die wählbaren Parameter und deren Einfluss auf das Ergebnis im Rahmen einer Sensitivitätsanalyse abzuschätzen. Darauf aufbauend wird ein standardisiertes Vorgehen zur Schaffung einer einheitlichen Datenbasis abgeleitet. Diese findet Anwendung im darauffolgenden Kapitel, in welchem das Tool genutzt wird, um die Verkehrs- und insbesondere die Stausituation verschiedener Siedlungsgebiete zu analysieren und daraus für diese einen vergleichbaren Stauindex zu generieren. 
In einem abschließenden Kapitel werden die Ergebnisse der Seminararbeit in einem Fazit zusammengefasst und deren Nutzen und Anwendbarkeit für die Verkehrsplanung sowie für Mobilitätsunternehmen erläutert.

